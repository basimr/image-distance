% !TEX TS-program = pdflatex
% !TEX encoding = UTF-8 Unicode

% This is a simple template for a LaTeX document using the "article" class.
% See "book", "report", "letter" for other types of document.

\documentclass[11pt,twocolumn]{article} % use larger type; default would be 10pt

\usepackage[utf8]{inputenc} % set input encoding (not needed with XeLaTeX)

%%% Examples of Article customizations
% These packages are optional, depending whether you want the features they provide.
% See the LaTeX Companion or other references for full information.

%%% PAGE DIMENSIONS
\usepackage{geometry} % to change the page dimensions
\geometry{a4paper} % or letterpaper (US) or a5paper or....
% \geometry{margin=2in} % for example, change the margins to 2 inches all round
% \geometry{landscape} % set up the page for landscape
%   read geometry.pdf for detailed page layout information

\usepackage{graphicx} % support the \includegraphics command and options

% \usepackage[parfill]{parskip} % Activate to begin paragraphs with an empty line rather than an indent

%%% PACKAGES
\usepackage{booktabs} % for much better looking tables
\usepackage{array} % for better arrays (eg matrices) in maths
\usepackage{paralist} % very flexible & customisable lists (eg. enumerate/itemize, etc.)
\usepackage{verbatim} % adds environment for commenting out blocks of text & for better verbatim
\usepackage{subfig} % make it possible to include more than one captioned figure/table in a single float
% These packages are all incorporated in the memoir class to one degree or another...

%%% HEADERS & FOOTERS
\usepackage{fancyhdr} % This should be set AFTER setting up the page geometry
\pagestyle{fancy} % options: empty , plain , fancy
\renewcommand{\headrulewidth}{0pt} % customise the layout...
\lhead{}\chead{}\rhead{}
\lfoot{}\cfoot{\thepage}\rfoot{}

%%% SECTION TITLE APPEARANCE
\usepackage{sectsty}
\allsectionsfont{\sffamily\mdseries\upshape} % (See the fntguide.pdf for font help)
% (This matches ConTeXt defaults)

%%% ToC (table of contents) APPEARANCE
\usepackage[nottoc,notlof,notlot]{tocbibind} % Put the bibliography in the ToC
\usepackage[titles,subfigure]{tocloft} % Alter the style of the Table of Contents
\renewcommand{\cftsecfont}{\rmfamily\mdseries\upshape}
\renewcommand{\cftsecpagefont}{\rmfamily\mdseries\upshape} % No bold!

%%% END Article customizations

%%% The "real" document content comes below...

\title{My Assignment Title}
\author{My Name}
%\date{} % Activate to display a given date or no date (if empty),
         % otherwise the current date is printed 

\begin{document}
\maketitle

\section{Introduction}

Lorem ipsum dolor sit amet, consectetur adipiscing elit. Sed sapien mi, vulputate eget volutpat sed, iaculis nec augue. Interdum et malesuada fames ac ante ipsum primis in faucibus. Fusce eget diam nisl. Duis enim felis, volutpat id lectus eget, rhoncus vehicula lacus. Aenean finibus augue vel urna sodales congue. Donec suscipit fermentum velit sed egestas. Curabitur eget tellus pulvinar ante ornare accumsan. Etiam feugiat, elit a congue dictum, orci orci pulvinar dui, ac porttitor tortor magna nec massa. Suspendisse bibendum dui purus, ut placerat justo fermentum et. Vivamus eget ex ultrices, luctus magna eget, tristique nibh. Phasellus luctus quis neque a viverra.

\section{Dijkstra's Algorithm}

Donec ut augue mattis, consectetur ligula luctus, maximus leo. Nam auctor vitae purus et egestas. Ut eu porttitor lorem. Phasellus ac scelerisque augue. Nam quis est vitae nunc faucibus sodales. Ut posuere dignissim urna semper dignissim. Proin nec eleifend nunc. Nunc finibus ligula nec fringilla ultricies. In tempor mauris arcu, in tristique odio ultrices at.

\subsection{Stale Nodes vs. Decrease-Key}

Curabitur dapibus mauris urna, vitae tincidunt dui consequat non. Maecenas orci arcu, venenatis eget suscipit at, malesuada sed velit. Integer cursus tortor sed placerat rhoncus. Aliquam id dolor at nisl pellentesque tincidunt. Quisque ullamcorper, mauris eget luctus semper, diam mi viverra sapien, vitae viverra tellus libero nec diam. Pellentesque habitant morbi tristique senectus et netus et malesuada fames ac turpis egestas. Aliquam viverra nisi quis sem luctus, laoreet tristique sapien pellentesque.

\section{Results}

Aenean condimentum felis diam, ut mollis quam lacinia ac. Proin et elit non est porttitor ultrices. Aenean suscipit dictum eros vel cursus. Ut eget malesuada metus, sit amet dictum velit. Nunc eleifend quam sed interdum venenatis. Proin vitae nisl et enim iaculis malesuada sit amet id augue. Fusce sagittis nibh sit amet hendrerit maximus. Curabitur porttitor sapien feugiat, lacinia urna suscipit, vestibulum nulla. Vivamus hendrerit blandit massa sit amet pretium.

\section{Discussion}

Phasellus sit amet lectus scelerisque mi efficitur egestas. Nullam pellentesque mattis quam ac fermentum. Praesent non urna lacus. Vivamus ac vulputate dolor, vitae convallis sem. Cras id lorem sed lorem condimentum tempor. Donec vulputate lorem et velit faucibus, ac elementum libero consequat. Proin imperdiet interdum libero, vel molestie risus vulputate non. Mauris sit amet dui pulvinar, gravida dolor id, efficitur neque. Sed sem velit, placerat eget erat id, fringilla aliquet est. Phasellus tortor tortor, lacinia quis mollis quis, ultrices eget neque. Nam vel scelerisque turpis. Maecenas tempor mollis nisi vitae efficitur.

\section{Conclusion}

The end!

\end{document}
